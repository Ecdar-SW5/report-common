\subsubsection{The Mutual Definition of Done}\label{common:collaboration:DoD}
To better facilitate multiple Scrum teams working on the same codebase, a Definition of Done (DoD) is necessary. The objective of defining a DoD is to have a shared agreement of when a feature is done, this shared definition was described in the first Scrum of Scrums meeting.

The definition we settled on was that every pull request to the GUI- and Reveaal-repositories' main-branches should be reviewed internally in the team as a draft, before it is sent out for an external review.
The external review consists of four different checks:
\begin{itemize}
    \item Ensure the new functionality described is implemented and works.
    \item Run a benchmark on the main- as well as the new functionalities branch.
    \item Check if the new code is properly documented. The I/O and panic criteria should be documented as well as what it does if functions are public.
    \item All tests should pass and if there is a new functionality it should be tested. It is up to the reviewers to judge if the tests are good enough.
\end{itemize}
Secondly the new functionality should be approved by two external reviewers.
%The external review consists of two people from other teams that follow a defined review guide which includes checks on functionality, performance, documentation, and tests.
If none of these areas are found lacking, and both reviewers approve of the changes, it is considered done.

It is worth noting that we have not made a formal definition for anything other than the code. However a review process for the collaborative part of the report exists. The review process is threefold, each Scrum team separately reviews the report. Thereafter each teams is assigned an area of responsibility, which is corrected and proof read. Lastly the original person/team who commented on the now corrected part, either accepts the change, or contacts the person who made it. After this process, the collaborative writing is considered done.
Any other DoD will be at the individual team's discretion to create.


% The mutual definition of DoD between the Scrum groups is:
% \begin{itemize}
%     \item Each increments should be reviewed by two students from the Scrum of Scrums.
%     \item Increments shall reviews internally in the group before making a pull request on Github.
%     \item If a increment is not reviewed in less than a day, the groups have failed to uphold their commitment to review other pull requests. 
%     \item Write code documentation that fits the written code. Such that having i/o variables defined, panic criteria and what the function descriptions.
%     \item It is up to the reviewers of the pull requests to say if the code is tested enough. 
% \end{itemize}