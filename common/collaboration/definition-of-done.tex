\subsubsection{The mutual definition of done}
During this project all teams will be working on the same codebase, which means it is most effective if we agree upon when a feature is done.
A shared DoD was therefore made during the first scrum of scrums meeting.
The definition we settled on was that every pull request to the GUI- and Reveeal-repositories' mains should be reviewed internally in the team as a draft, before it is sent out for an external review.
The external review consists of two people from other teams that follow a defined review guide which includes checks on functionality, performance, documentation, and tests. 
If none of these areas are found lacking, and both reviewers approve of the changes, it is considered done.

It is worth noting that we have not made a formal definition for anything other than the code. For the parts of the report that has been written collectively it will be considered done when none of the teams have anything more to add.
Any other DoD will be at the individual teams discretion to create.


% The mutual definition of DoD between the Scrum groups is:
% \begin{itemize}
%     \item Each increments should be reviewed by two students from the Scrum of Scrums.
%     \item Increments shall reviews internally in the group before making a pull request on Github.
%     \item If a increment is not reviewed in less than a day, the groups have failed to uphold their commitment to review other pull requests. 
%     \item Write code documentation that fits the written code. Such that having i/o variables defined, panic criteria and what the function descriptions.
%     \item It is up to the reviewers of the pull requests to say if the code is tested enough. 
% \end{itemize}