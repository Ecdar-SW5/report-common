\subsection{Forks}
As mentioned in Section \ref{sub:shared-tools-for-collaboration}, the Scrum groups will utilize GitHub for version management of the code.
However, the groups will not be directly altering the original "upstream" repositories of ECDAR.
Instead, the groups will fork the repositories of the original ECDAR project. Forking repositories means that new repositories will be created that share the code and visibility settings with the upstream repositories of the original project \cite{github_fork}.
This strategy is used, since the changes to ECDAR that the groups are proposing have to be approved by the developers that are in charge of the ECDAR project, and it is therefore not desired that the groups proposals are directly merged into the upstream repositories.
At the end of the semester, the developers in charge will look at the proposed ideas, and it is up to them to decide what should be kept for the upstream repositories.
It should therefore be noted, that not everything implemented by the groups during this semester will necessarily be a part of ECDAR.  