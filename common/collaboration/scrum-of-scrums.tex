\subsection{Scrum of Scrums}\label{sub:scrum-of-scrums}
As described in the previous section we have elected to use Scrum of Scrums. Scrum of Scrums is the implementation of a Scrum framework on top of multiple groups working with the Scrum method, essentially creating a high-level Scrum team. Each Scrum team should therefore elect a member to attend the Scrum of Scrums meetings.

%In scrum of scrums we do not have daily scrums because the daily scrums is held in the smaller groups.
The Daily Scrum event is only used by each individual group, and not by the overarching Scrum of Scrums. This change has been made, as each group works on independent features and thus a Daily Scrum between the groups is unnecessary for the given project.
%Working in scrum of scrums allows us as an organization to only hold daily scrum in each individual group, rather than across the entire organization.
%This simplifies the entire work process, ensuring focus. 
The key events in Scrum of Scrums are, like they are in Scrum, the Sprint Planning, Sprint Review, and Sprint Retrospective. These events are handled by representatives of each group, ensuring that the collaborative processes between groups are agreed upon.
See \autoref{fig:scrum-of-scrums-events} for a more detailed look of the different events between Scrum and Scrum of Scrums.

To allow for short and concise Scrum of Scrums meetings, each group should hold a meeting internally after the Sprint planning and before the review and retrospective. This ensures the representative is ready for the main Scrum of Scrums meeting. This process is also described in \autoref{fig:scrum-of-scrums-events}.

%sending their committee to the main meeting for the scrum of scrums. 
%Based on that the groups choose a committee to attend the scrum meetings.

\begin{figure}[H]
    \centering
    \begin{tikzpicture}[font=\bfseries\scriptsize]
    \node(daily) [process] {Daily Scrum Meetings};
    \node(planL) [process, left=1cm of daily] {Sprint Planning};
    \node(planU) [process, above=1cm of planL, xshift=-.5cm] {Sprint Planning};
    \node(revL) [process, right=1cm of daily] {Sprint Review\\+\\Sprint Retrospective};
    \node(revU) [process, above=1cm of revL, xshift=.5cm] {Sprint Review\\+\\Sprint Retrospective};
    
    \node [left=2cm of planL, text width=.7cm] {Individual Group Meetings};
    \node [left=1.5cm of planU, text width=.7cm] {Group Representatives Meeting};
    
    \node(time) [below=1.5cm of daily] {\normalsize Time};
    \node(w1) [above=.15cm of time] {Week 1};
    \node(w0) [left=3cm of w1] {Week 0};
    \node(w2) [right=3cm of w1] {Week 2};
    
    \draw   (daily) edge[arrow] (revL)
            (revL) edge[arrow] (revU)
            (revU.north) edge[arrow, above, bend right] (planU.north)
            (planU) edge[arrow] (planL)
            (planL) edge[arrow] (daily);
    \draw (w0.north)++(left: .55cm)++(up: .2cm) edge[arrow, above] ++(right: 9.75cm);
    \end{tikzpicture}
    \caption{A figure depicting the events in a scrum sprint, divided between the big Scrum of Scrums group and the smaller Scrum groups.}
    \label{fig:scrum-of-scrums-events}
\end{figure}


As mentioned, the multi-project focuses on both the GUI and Reveaal engine.
Because of this dual focus, we have two Product Owners: one for the Reveaal teams and one for the GUI team.
The project's different Product Owners will help uphold the Product Backlog and update it if any changes appear. 
These changes are discussed during the Scrum of Scrums meetings, where the Product Owners can also be present.

The length of a Sprint is measured against the length of the smaller Scrum group's Sprints.
Because of this, the average Sprint length is two weeks.
Based on the length of the Sprint, the Scrum of Scrums group has two meetings every two weeks. 
One meeting is for planning the upcoming Sprint, and one is for Sprint Reviewing and Retrospective at the end of a Sprint.
These meetings, if possible, are held on the first Monday of a Sprint, and the last Friday of a Sprint, respectively.

% One for planing the upcoming sprint the first monday of the sprint, if possible and one meeting on last friday of the sprint, if possible.
% The last meeting's topics are always sprint reviewing and sprint retrospecting. 



