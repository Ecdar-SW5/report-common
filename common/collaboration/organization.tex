\subsection{The ECDAR-SW5 Organization}\label{sub:ecdar-organization}
There must be some kind of organization in order to orchestrate collaboration between all the project groups that are working on the ECDAR project. 
Organization is important because it establishes channels of communication, a distribution of the workload, and a process for evaluating the outcome of each stage in the development process.

All of the project groups that are a part of the ECDAR-SW5 organization are themselves a scrum team. 
As scrum is dependent on the teams being small, the organization must use an agile scaling framework. 
Such a framework ensures distribution of the workload between the teams and that the output from each team can be integrated. 

% When an organization needs to pick a framework to implement, it must consider its needs. 
% The groups that make up the ECDAR-SW5 organization must be self-managed and free to decide how they work internally. 
% They have responsibility for their own project after all. 
% The organization has to choose a framework that allows teams this flexibility.
% This is why we have chosen to work in "scrum of scrums".

As an organization, the ECDAR-SW5 organization found it crucial to consider its needs before picking a framework to implement.
Due to the requirements from Aalborg University as well as what is preferred within the six individual groups, each group must be self-managed and free to decide how they work internally.
This led us to choose to work in "scrum of scrums", allowing each group to work in their own way without sacrificing the teamwork that scrum provides across the organization.

%As agile is very popular within some of the largest organizations in the world(SOURCE PLEASE), multiple different agile scaling frameworks have been invented. 
