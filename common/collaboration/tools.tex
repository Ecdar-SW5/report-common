\subsection{Shared tools for collaboration}\label{sub:shared-tools-for-collaboration}
For every group in the scrum groups, there is a shared toolbox.
This toolbox contains tools that help the groups cooperate with each other.
The main tool is GitHub and it is used for version management.  
Each group has their own repository, minimizing the risk of groups getting in the way of each other's work.
Wherever two groups may need to share a repository, they maintain individual branches.
Because of this, a branching strategy was made during the first scrum of scrums meeting.
For each increment, a branch is made from the main branch on the specific repository. 
When an increment is finished, tested, and reviewed, the branch can be merged back into the main branch.

To keep track of increments and the backlog, a project board is used on GitHub.
This board holds the information of every different increment that the individual groups work on.

To communicate between groups, a discord server was created.
The server contains different text channels that are used for different purposes, e.g a channel for collaborating on writing the joint sections and a channel for setting up meetings.

For every meeting in the scrum of scrums, an agenda is created and distributed between the groups through OneDrive.
OneDrive is also used for sharing documents, notes, etc.

The joint writing is done in Overleaf and later uploaded to GitHub in a repository.
This makes it possible to write together in real time and later insert the sections into the individual groups' reports.
