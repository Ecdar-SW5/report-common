\subsection{Shared tools for collaboration}\label{sub:shared-tools-for-collaboration}
For every group in the scrum groups there are a shared toolbox.
This toolbox contains tools that helps the groups cooperate between each other.
The main tool is Github and it is used for version management.  
Each group have their own repository and if the groups shares the same repository then they have individual branches.
Because of this, a branching strategy was made on the first meeting in the scrum of scrums.
For each increment a branch is made from the main branch on the specific repository. 
When an increment is done, tested, and reviewed the branch can be merged back into the main branch.

To keep track of increments and the backlog, a project board is used on Github.
This board holds all the information of the different increments, the individual groups works on.

To communicate between groups a discord server was created.
The server contains different text channels that are used for different purposes e.g a channel for collaborating on writing the joint sections and a channel for setting up meetings.

For every meeting in the scrum of scrums an agenda is created and distributed between the groups through a OneDrive.
The OneDrive is also used for sharing documents, notes, etc..

The joint writing is done in Overleaf and later uploaded to Github in a repository.
This makes it possible to write together in real time and later insert the sections into the individual group's reports.
This concludes the section of the shared tools.