\subsection{Shared Tools for Collaboration}\label{sub:shared-tools-for-collaboration}
All Scrum groups share a common toolbox.
This toolbox contains tools that help the groups cooperate with each other.
The primary tool is GitHub, and it is used for version management.  
Each group has its own repository, minimizing the risk of groups getting in the way of each other's work.
Wherever two groups may need to share a repository, they maintain individual branches.
Because of this, a branching strategy was made during the first Scrum of Scrums meeting.
For each Increment, a branch is made from the main branch on the specific repository. 
When an Increment meets the DoD, the branch can be merged back into the main branch.

To keep track of Increments and both the Product- and the Sprint Backlog, a project board is used on GitHub.
This board holds the information on every different Increment that the individual groups work on.

Another tool in the toolbox is Discord.
A Discord server was created and used for communication between the groups.
The server contains different text channels that are used for different purposes, e.g a channel for collaborating on writing the joint sections, and a channel for setting up meetings.

OneDrive is another tool in the common toolbox.
For every meeting in the Scrum of Scrums, an agenda is created and distributed between the groups through OneDrive.
OneDrive is also used for sharing documents, notes, etc.

The joint writing is done in Overleaf and later uploaded to GitHub in a repository.

Overleaf makes it possible to write together in real time and later insert the sections into the individual groups' reports, since all the groups write their reports in overleaf.

