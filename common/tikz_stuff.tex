\usetikzlibrary{shapes.geometric, arrows, fit, calc, automata, positioning, arrows.meta, decorations.markings}

\tikzstyle{io} = [trapezium, trapezium left angle=70, trapezium right angle=110, minimum width=3cm, minimum height=1cm, text centered, draw=black, fill=white!30]
\tikzstyle{decision} = [diamond, text centered, draw=black, fill=white!30, aspect=2]
\tikzstyle{block_frame} = [dashed, line width=1pt, draw=black,inner sep=0.4em]

\newcommand\coordSys[2]{
    \draw[step=1cm,gray, thick] (-.25,-.25) grid (#1.9,#2.9);
    \draw[thick,->] (0,0) -- (#1.5,0) node[anchor=north west] {$x$};
    \draw[thick,->] (0,0) -- (0,#2.5) node[anchor=south east] {$y$};
    \foreach \x in {0,...,#1}
       \draw (\x cm,1pt) -- (\x cm,-1pt) node[below=.2cm] {$\x$};
    \foreach \y in {0,...,#2}
        \draw (1pt,\y cm) -- (-1pt,\y cm) node[left=.2cm] {$\y$};
}

\newcommand\drawDBM[2]{\filldraw[fill=#2, draw=black] foreach \p in #1 {-- \p} -- cycle;}




%%% SU-shapes and figures
\def\bottom#1#2{\hbox{\vbox to #1{\vfill\hbox{#2}}}}
\tikzstyle{process} = [rectangle,rounded corners, minimum width=2cm, minimum height=1cm, text centered, text width=1.85cm, draw=black, fill=white!30,]
%\tikzstyle{state} = [rectangle, minimum width=3cm, minimum height=1cm, text centered, text width=3cm, draw=black, fill=white!30, rounded corners=14pt]
\tikzstyle{case} = [ellipse, minimum width=1cm, minimum height=1.2cm,
text centered, draw=black, fill=white!30, node distance=1.3cm and 0cm, text width=2.5cm, inner sep=-.2cm]
\tikzset{actor/.style={
    label={[yshift=-.5cm]north:{\centering\textbf{$\ll$actor$\gg$}}},
    process, anchor=south, node distance=5cm, text width=2.2cm
}}

% arrows
\tikzstyle{arrow} = [thick,->,>=Stealth]
\tikzstyle{gen_arrow} = [thick,->,>={Triangle[open, scale=2.25]}]
\tikzstyle{aggregation} = [thick,->,>=]
\tikzstyle{event_arrow} = [thick,->,>={Straight Barb[scale length=3]}]
\tikzset{-<>/.style={
    decoration={
        markings,
        mark= at position 1 with {\arrow{Diamond[open, scale=2, scale width=1.5]}},
    },
    postaction={decorate},
    shorten >=8pt,
}}
% Events
\tikzstyle{event_start} = [draw,circle,scale=.75,fill=black]
\tikzset{
  event_end/.style={
    label={[draw,circle,scale=1.3,line width=0.4mm]center:},
    label={[draw,circle,scale=.75,fill=black]center:}
  }
}
%\tikzstyle{state} = [rectangle, minimum width=1.5cm, minimum height=1cm, text centered, draw=black, fill=white!30]
% cluster
\tikzset{cluster/.style={
    label={[fill=white, draw=black, line width=.75pt, minimum width=1.65cm, minimum height=0.5cm, text width=1.65cm,
    xshift=54.4pt, yshift=-0.7pt]north west:{\centering$\ll$cluster$\gg$\\}#1},
    draw, line width=.75pt, inner sep=1em,
}}

% component
\tikzset{comp/.style={
    draw, line width=.75pt, inner sep=1em,
    label={[fill=white, draw=black, line width=.75pt, minimum width=1.65cm, minimum height=0.5cm, text width=2cm,
    xshift=62pt, yshift=-0.6pt, inner sep=.2em, anchor=south east]north west:{\centering\textbf{$\ll$component$\gg$\\#1}}},
}}
\tikzset{compS/.style={
    label={[fill=white, draw=black, line width=.75pt, minimum width=1cm, minimum height=0.3cm,
    xshift=29.3pt, yshift=-0.7pt]north west:},
    minimum width=2.5cm, minimum height=.5cm, text centered, text width=2.1cm, draw, line width=.75pt, inner sep=.5em, font={\scriptsize\bfseries$\ll$component$\gg$\xspace} %label={[text centered, text width=2cm]center:{$\ll$component$\gg$\\#1}},
}}

%% Class with atrr, and func
% To use , simply write \class{name}{attributes}{funcitons} in a tikzpicture
% It can be refered to as the 'name'
\newcommand\class[4][]{
    \begin{scope}[node distance=0cm,local bounding box=#2, #1]
        \node(name) [process, minimum height=.5cm, minimum width=3cm, text width=2.7cm] {\textbf{#2}};
        \node(attr) [process, minimum height=.5cm, minimum width=3cm, align=left, yshift=1pt, text width=2.7cm, below=0mm of name] {#3};
        \node(func) [process, minimum height=.5cm, minimum width=3cm, align=left, yshift=1pt, text width=2.7cm, below=0mm of attr] {#4};
    \end{scope}
}


\makeatletter
\newcount\dirtree@lvl
\newcount\dirtree@plvl
\newcount\dirtree@clvl
\def\dirtree@growth{%
  \ifnum\tikznumberofcurrentchild=1\relax
  \global\advance\dirtree@plvl by 1
  \expandafter\xdef\csname dirtree@p@\the\dirtree@plvl\endcsname{\the\dirtree@lvl}
  \fi
  \global\advance\dirtree@lvl by 1\relax
  \dirtree@clvl=\dirtree@lvl
  \advance\dirtree@clvl by -\csname dirtree@p@\the\dirtree@plvl\endcsname
  \pgf@xa=1cm\relax
  \pgf@ya=-0.45cm\relax
  \pgf@ya=\dirtree@clvl\pgf@ya
  \pgftransformshift{\pgfqpoint{\the\pgf@xa}{\the\pgf@ya}}%
  \ifnum\tikznumberofcurrentchild=\tikznumberofchildren
  \global\advance\dirtree@plvl by -1
  \fi
}
\tikzset{
  dirtree/.style={
    growth function=\dirtree@growth,
    every node/.style={anchor=north},
    every child node/.style={anchor=west},
    edge from parent path={(\tikzparentnode\tikzparentanchor) |- (\tikzchildnode\tikzchildanchor)}
  }
}