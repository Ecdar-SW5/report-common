\section{Introduction to ECDAR}\label{sec:introduction-to-ecdar}
\commondisclaimer

This section serves as an overview over the different concepts of ECDAR to provide a basic understanding of what ECDAR is, the purpose of ECDAR, its architecture, technologies behind it, and what state ECDAR is currently in.

ECDAR stands for \textbf{E}nvironment for \textbf{C}ompositional \textbf{D}esign and \textbf{A}nalysis of \textbf{R}eal Time Systems.
ECDAR is a graphical tool to model real-time systems using timed input/output automata, and analyse these systems. 
This process is called model checking. ECDARs website can be found on \href{https://www.ecdar.net/}{https://www.ecdar.net/}.

%what
Model checking is a method of verifying that models in the form of finite-state automata fulfill specified criteria \cite{modelchecking-handbook}. 
An automaton is defined as one or more components, but a component is also an automaton.
The bigger an automaton becomes, the harder it becomes to verify it due to increased complexity. 
Having computer-aided checking ensures that all edge cases are checked, where one can be forgotten if done manually.
 
%why
% The importance of a tool like ECDAR is best described by an example:
% Imagine you are working for a space program and want to send a satellite into space.
% You spent several months testing the model for the satellite to make sure it is correct and everything is working as intended.
% The day after launch, a major problem in the system is found. However since the satellite is already in space, it is not possible to just update it as easily as other systems. If instead a tool like ECDAR had been used, the issue would have been identified before the satellite was launched, thereby allowing the engineers to solve the problem.
% This is why ECDAR is a useful tool to model check systems: to verify that a model's components works together as expected.

Imagine that you are working for a space agency and your mission is to send a satellite into space.
You have spent several months testing the satellite to make sure it is correct and everything is working as intended.
All tests are passed for every edge case that your engineers could think of.
The satellite is ready to be launched.
However, the day after launch, a major problem in the system is found. 
It will need to be fixed if the satellite is to have any use, but doing so might be extremely expensive or outright impossible.

As the model grows in size, it becomes ever more complex to verify by hand.
The immense complexity of modern systems dictates that the verification must be computer-aided to be feasible \cite{modelchecking-handbook}. ECDAR is one such tool.

It is important to model check a system to ensure that it behaves as intended.
In the above example with the faulty satellite, there are several interdependent components, which ECDAR is able to take into account.
If one of these components fails or is blocked, then it may make the entire satellite faulty.
By providing a tool that can be reliable at testing if a new system is true to a given specification, we can not only ensure correctness but also efficiency in comparison to doing the check by hand.

Model checking as a tool can try to guide the user as much as possible in accordance to the specification, but in the end, the user can still make a wrong specification. Therefore model checking can only ensure that the system works in the real world if the specifications are correct.