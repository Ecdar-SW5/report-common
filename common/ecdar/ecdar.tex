\section{Introduction to ECDAR}\label{sec:introduction-to-ecdar}
\commondisclaimer

This section serves as an overview over the different concepts of ECDAR to provide a basic understanding of what ECDAR is, the purpose of ECDAR, its architecture, technologies behind it, and what state ECDAR is currently in.

ECDAR stands for \textbf{E}nvironment for \textbf{C}ompositional \textbf{D}esign and \textbf{A}nalysis of \textbf{R}eal Time Systems.
ECDAR is a graphical tool to model real-time systems using timed input/output automata, and analyse these systems. 
This process is called model checking. Ecdars website can be found on \href{https://www.ecdar.net/}{https://www.ecdar.net/}.

%what
Model checking is a method of verifying that models in the form of finite-state automata are correct \cite{modelchecking-handbook}. A model is defined as one or more components, but a component is also a model.
The bigger a model becomes, the harder it becomes to verify it. 
Having computer-aided checking ensures that all edge cases are checked, where one can be forgotten if done manually.
 
%why
The importance of a tool like ECDAR is best described by an example:
Imagine you are working for a space program and want to send a satellite into space.
You spent several months testing the model for the satellite to make sure it is correct and everything is working as intended.
The day after launch, a major problem in the system is found. However since the satellite is already in space, it is not possible to just update it as easily as other systems. If instead a tool like ECDAR had been used, the issue would have been identified before the satellite was launched, thereby allowing the engineers to solve the problem.
This is why ECDAR is a useful tool to model check systems, to verify that a model's components works together as expected.


%Model checking is usually associated with real time hardware or systems, since validity of the life cycle needs checking.
%There is also a need for certain safety requirements to ensure that there are no crashes or other obstacles throughout the life cycle.

It is important to model check a system to ensure that it behaves as intended.
In the above example with the faulty satellite, there are several interdependent components.
If one of these components fails or is blocked, then it makes the entire satellite faulty.
% Take for instance the satellite example, it is a very complex hardware/system with a lot of functionality.
% The satellite itself has several components, which are dependent on each other.
By providing a tool that can be reliable at testing if a new system is true to a given specification, we can not only ensure correctness but also efficiency.


Model checking as a tool can try to guide the user as much as possible in accordance to the specification, but in the end, the user can still make a wrong specification. Therefore model checking can only ensure that the system works in the real world if the specifications are correct.
%Also how efficient ECDAR is as a tool is dependent on what kind of algorithms are used, and the efficiency of the run-time.

%%ECDAR therefore tests the correctness of the model using timed I/O automata.
%% It parallelizes test-case generation and test execution to provide this significant speed-up.

%To sum up, what the purpose of ECDAR is:
%\begin{quote}
%"[to integrate] conformance testing into a new IDE that now features modelling, verification, and testing. The new tool uses model-based mutation testing, requiring only the model and the system under test, to locate faults and to prove the absence of certain types of faults." \cite{Gundersen_2018}
%\end{quote}



