\subsection{Scrum}
Scrum is an agile framework that developers use to ensure a development process where "Commitment, Focus, Openness, Respect, and Courage" is valued and practiced \cite{schwaber_sutherland_2022}. According to the scrum framework, a \emph{Scrum Master} is responsible for making sure the team is focused on reaching the goal and maintaining an environment where the scrum values are upheld. A \emph{Product Owner} is responsible for creating and maintaining a \emph{Product Backlog}, an ordered list of the work that needs to be completed. Finally, the \emph{Scrum Team} is responsible for planning and creating the \emph{Increments} that will further the development towards the product goal. 


% Roles
\subsubsection{Roles}
Scrum is based on having a small and nimble team that is still large enough to make significant progress each sprint. 
These teams should typically be less than 10 people.
Members of scrum teams are assigned 3 roles. \cite{schwaber_sutherland_2022}

\paragraph{Product Owner}
In a team there should be one product owner. However it's also possible to share the same product owner across multiple scrum teams if they are focused on the same product.
The product owner is responsible for maximizing the value created by the scrum teams. It is also the product owners responsibility to manage the product backlog and communicating what the product goal is. \cite{schwaber_sutherland_2022}

\paragraph{Scrum Master}
Apart from the product owner there should also be one of the developers who take on the role as a scrum master.
The scrum master is mainly responsible for implementing and making the team follow the scrum framework, which should improve the teams effectiveness. \cite{schwaber_sutherland_2022}
% There is a lot more responsibilities that the scrum master has, which are related to serving the team, product owner and organization 

\paragraph{Developers}
The rest of the team members are all developers and are accountable for creating usable increments each sprint.
They take part in creating the sprint backlog, ensuring that increments adhere to the definition of done and adapting their daily plan in accordance with the current sprint goal. \cite{schwaber_sutherland_2022}

% Events
\subsubsection{Scrum Events}
Scrum contains 5 events, the primary event is the sprint, in which all other events are contained. The other events are sprint planning, daily scrum, sprint review and sprint retrospective.
These events are designed to enable transparency and lessen the need for meetings that are not defined within scrum. \cite{schwaber_sutherland_2022}

\paragraph{The Sprint}
The primary sprint event is a fixed length event typically a month or less. This event is where value is created.
All the other scrum events are completed during the sprint.
...\cite{schwaber_sutherland_2022}

\paragraph{Sprint Planning}
The first event in a sprint is typically the sprint planning, which is plan that defines the work that's supposed to be done during the sprint.
This plan is created as a collaborative effort by the scrum team.
...

\paragraph{Daily Scrum}
Daily scrum is a short meeting that's completed each day where the progress is inspected 


\subsubsection{Scrum Artifacts}
The product backlog is one of the three artifacts in scrum and the progress of the product backlog is compared to the product goal.
It's an ordered list that indicated which components needs to be improved.
The list used during the sprint planning, where the scrum groups to chooses the work for the upcoming sprint.
Refinement of the product backlog is a breakdown of the list and further defining the product backlog items into smaller and more precise items.
This is a ongoing event throughout the different sprints and this adds more detail to the product backlog in the end. \cite{schwaber_sutherland_2022}

The second artifact is the sprint backlog. 
This artifact is the set of product backlog items the scrum group has selected for the sprint. 
This list if items from the product backlog is planned by and for the developers in the scrum teams. 
Throughout the sprint the print backlog is updated each time the scrum group have learned something new. \cite{schwaber_sutherland_2022}

The increment is the third and last artifact in scrum and it is a concrete stepping stone towards the product goal.
Every increment is thoroughly verified to ensure that all the increments work together because each increment is additive to the prior increments the scrum teams have completed.
It is possible that multiple increments is created within a sprint.
An increment cannot be considered as work if it do not uphold the criterion in the definition of done. \cite{schwaber_sutherland_2022}
