\section{Agile Software Development} \label{sec:agile-software-development}
This section gives an introduction to agile development in general, scrum will also be introduced as a specific agile framework. Planned development through the waterfall method will also briefly be introduced, as an alternative to the agile approach.
Before agile development, project management relied heavily on the waterfall method \cite{sommerville}. 
Waterfall entails planning everything before proceeding with development in stages.
Every stage is completed entirely before the next begins. 
This method can be useful when both the problem and implementation is clearly defined and unlikely to change during process. 
The problem with waterfall is that it has no mechanism for handling uncertainty \cite{sommerville}. 
Because we are working with a codebase, language as well as techniques that are completely new to most of us, we wish to stay open in our work strategy, so as to avoid getting stuck.
If the requirements change during development, the waterfall method cannot account for that.
%As it is possible for requirements to change during development, the waterfall model is not ideal for this particular project.
Agile development practices arose from a need for a development method that would solve the issues posed by waterfall.
Most importantly, it would lessen the cost of making changes to the specification late in the development process \cite{alancline}. 
The Agile Manifesto from 2001 outlines the values that agile development practices are supposed to support \cite{beck2001agile}. 
The first statement of the manifesto is: \say{Individuals and interactions over processes and tools} \cite{beck2001agile}.
Agile development is ideal for projects that require flexibility and communication, which makes it ideal for the Ecdar project. 
A potential disadvantage of agile development is lack of focus on documentation, which is something we must then be aware of throughout the project.
This statement concisely encompasses the change of focus from planning and strict application of well-defined procedures. 
Instead, the focus should be on delivering software that the user wants.
